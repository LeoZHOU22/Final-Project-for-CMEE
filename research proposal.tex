\documentclass{article}
\usepackage{graphicx} % Required for inserting images
\usepackage[natbibapa]{apacite}
\usepackage{natbib}
\usepackage{setspace}
\usepackage{lineno}

\begin{document}

\begin{titlepage}
    \begin{center}
        \vspace*{1cm}
        
        \Large
        \textbf{Project Proposal}
        
        \vspace{1.5cm}
        
        \textbf{Yutao ZHOU}\\
        Affiliation: Imperial College London\\
        Email: lz723@ic.ac.uk

        \vspace{0.8cm}

        \textbf{Supervisor}\\
        Sammraat Pawar, Imperial College London
       
        \vfill
        \Large
        Date:  30 Mar 2024
        
    \end{center}
\end{titlepage}

\linenumbers
\modulolinenumbers[5]
\noindent \textbf{Keywords: } Thermal performance curves(TPC), trait, rTPC(R package), methematical model 

\section{Introduction}
In the pursuit of understanding how organisms' physiological rates are influenced by temperature, Thermal Performance Curves (TPCs) have emerged as a cornerstone concept in ecological and evolutionary biology \cite{Kontopoulos2020}(more article). These curves, which depict the relationship between temperature and biological rates, are integral for predicting the impacts of climate change on biological systems from individual organisms to entire ecosystems \cite{Mordecai2019}(more). The complexity and diversity of TPC models necessitate robust computational tools for their analysis. \cite{Padfield2021} introduced a significant advancement in this arena with the development of the rTPC and nls.multstart R packages. This innovative computational pipeline facilitates the fitting of TPCs to large datasets, offering a selection from 24 distinct TPC models and addressing the urgent need for scalable and versatile tools in the face of burgeoning ecological and evolutionary data.

\hfill\break
However, the question of optimal model selection remains a formidable challenge, given the array of available models and their varying performance across different datasets. \cite{Kontopoulos2021} made strides in addressing this challenge by conducting a systematic comparison of 83 TPC models across a broad spectrum of traits and taxonomic groups. Their study highlighted the considerable variability in model performance and underscored the absence of a one-size-fits-all model for TPC analysis. This finding emphasizes the necessity of a thoughtful model selection process to mitigate potential biases and enhance the accuracy of TPC analyses. Integrating the insights from \cite{Kontopoulos2021} into the rTPC pipeline initiated by \cite{Padfield2021} offers a promising avenue for refining and expanding the toolkit available to researchers, thus paving the way for a deeper and more nuanced understanding of organisms' responses to temperature changes within the context of global climate dynamics.

\section{Proposed Methods}
\subsection{Data Prepossessing}

Utilizing tools from the rTPC package, we first standardize the preprocessing of ecological and evolutionary biology datasets, including adjustments to temperature ranges and outlier processing. This step ensures consistency across data, laying a solid foundation for subsequent analyses.

\subsection{Model Library Construction and Selection}

Drawing on the research by \cite{Kontopoulos2021}, we will build a more extensive library of TPC models, incorporating not only the 24 models found in rTPC but also other models validated within ecology and evolutionary biology. From this library, we develop a machine learning-based model selection algorithm capable of dynamically recommending the most suitable model based on the characteristics and taxonomic groups of the dataset.

\subsection{Adaptive Learning Algorithm}

We propose an adaptive learning algorithm that continually refines model recommendations using new data sources. By analyzing the performance of each model fitting, the algorithm updates the model's weights and preferences to achieve adaptive fitting to complex ecological data.

\subsection{Ensemble Learning and Model Integration}

An ensemble learning strategy is adopted, not to singularly select the “best” model but to integrate predictions from multiple models. Weights based on their historical performance enhance the overall prediction's accuracy and robustness.

\subsection{Model Validation and Application}

Statistical methods, including but not limited to cross-validation, AIC, and BIC, will be employed for model validation. Furthermore, we plan to demonstrate the application of this method in addressing real-world ecological and evolutionary problems through a series of case studies, such as predicting the impact of climate change on species distributions.
% References section
\section{Timeline}

\section{Budget}
no budget
\newpage  
\bibliographystyle{apacite}
\bibliographystyle{apalike} 
\bibliography{reference}
\end{document}
